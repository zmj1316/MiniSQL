\documentclass[notitlepage,cs4size,punct,oneside]{report}
\usepackage{CJK}
\usepackage{graphicx}
\usepackage{url}
\usepackage{listings}
\usepackage{enumitem}
\usepackage{xcolor}
\usepackage{txfonts}
\usepackage[bookmarks]{hyperref}  
\hypersetup{colorlinks=false,pdfborder=0 0 0}
% \renewcommand{\ttdefault}{Times New Roman}
\renewcommand{\ttdefault}{consolas}

\lstset{numbers=left,frame=none, numberstyle=\tiny, xleftmargin=.5em,xrightmargin=.5em, aboveskip=1em, escapeinside='',extendedchars=false ,
	keywordstyle=\small\bf\color{blue},
   % identifierstyle=\bf,
   numberstyle=\small\color{gray},
  commentstyle=\bf\it\small\color{green},
 stringstyle=\small\color{red},
 % lineskip=-2pt,
 % basicstyle=\ttfamily\small,
 language=C
 }

\begin{document}

\title{MiniSQL Document}
\author{Key Zhang}

\maketitle
\tableofcontents

\chapter{MiniSQL CLI}

\chapter{MiniSQL API}

\chapter{MiniSQL Structure}
	\section{Buffer Manager}
		\subsection{Data Struct}
		\paragraph{}
			\begin{lstlisting}
struct Buffer
{
    u8 win[BLOCKSIZE]; // The window for client to visit
    u32 winptr;// The block cached in win
    u8 dirty;
    u8 _cache[CACHESIZE][BLOCKSIZE];
    u32 _cacheptr[CACHESIZE];
    u8 _dirty[CACHESIZE];
};
typedef struct Buffer Buffer;
			\end{lstlisting}
		\subsection{Public Functions}
			\paragraph{}
			\begin{description}[align=left,style=sameline,leftmargin=7cm]
			\item [Func :] Description :
			\item [void buffer\_init(Buffer *, const char *);] Initialize the buffer with filename.
			\item [void sync\_window(Buffer *);] Synchronize the buffer window: Write back to cache if 	it's dirty.
			\item [void move\_window(Buffer *,u32);] Move the buffer window to specified block(	Synchronize First).
			\item [void extend(...)] Extend the file.
		\end{description}
	\section{Record Manager}
		\subsection{Data Struct}
		\paragraph{}
	    	\begin{lstlisting}
struct item
{
    dataType type;
    Data data;
};
typedef struct item item;

struct record
{
    item i[MAXCOLUMN];
    Bool valid;
};
typedef struct record record;	    		
	    	\end{lstlisting}
	    \subsection{Public Functions}
	    \paragraph{}
	    \begin{description}[align=left,style=sameline,leftmargin=8cm]
		\item [Func :] Description :
	    \item [MiniList *Recordmanager\_getRecord(table *tb);] Return all records in table.
	    \item [Insert]
	    \item [Select]
	    \item [Delete]
	    \item [...]
	    \end{description}
	\section{Catalog Manager}
		\subsection{Public Functions}
			\paragraph{}
			\begin{description}[align=left,style=sameline,leftmargin=7cm]
			\item [Func :] Description :
			\item [createTable]
			\item [connectTable]
			\item [dropTable]
			\item [RegIndex]
			\item [RMIndex]
		\end{description}
\end{document}
